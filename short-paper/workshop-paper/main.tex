\documentclass[conference]{IEEEtran}
\IEEEoverridecommandlockouts
% The preceding line is only needed to identify funding in the first footnote. If that is unneeded, please comment it out.
\usepackage{cite}
\usepackage{amsmath,amssymb,amsfonts}
\usepackage{algorithmic}
\usepackage{graphicx}
\usepackage{textcomp}
\usepackage{xcolor}
\def\BibTeX{{\rm B\kern-.05em{\sc i\kern-.025em b}\kern-.08em
    T\kern-.1667em\lower.7ex\hbox{E}\kern-.125emX}}

%=================================================================
%====ADDITIONAL PACKAGES & SETTINGS FROM THE ORIGINAL TEMPALTE====
\usepackage{import} % to use \import{../../../content-tex/}{abstract}
\usepackage{todonotes} % USAGE: \todo[inline]{is there a better example from X topic?}
\usepackage{lipsum}  % USAGE \lipsum[2-4]

\begin{document}

\title{

% Diversity, Equity, and Inclusion of Artificial Intelligence and Robotics for children %Fri  7 Jan 23:59:42 GMT 2022
Piloting Inclusive Workshops of Artificial Intelligence and Robotics for children %Sat  8 Jan 00:12:35 GMT 2022

    }

\author{

\IEEEauthorblockN{1\textsuperscript{st} AIR4Children}
\IEEEauthorblockA{\textit{dept. name of organization (of Aff.)} \\
\textit{air4children: Artificial Intelligence and Robotics}\\
Xicohtzinco, M\'exico \\
air4children@gmail.com}

}

% \author{\IEEEauthorblockN{1\textsuperscript{st} Given Name Surname}
% \IEEEauthorblockA{\textit{dept. name of organization (of Aff.)} \\
% \textit{name of organization (of Aff.)}\\
% City, Country \\
% email address or ORCID}
% \and
% \IEEEauthorblockN{2\textsuperscript{nd} Given Name Surname}
% \IEEEauthorblockA{\textit{dept. name of organization (of Aff.)} \\
% \textit{name of organization (of Aff.)}\\
% City, Country \\
% email address or ORCID}
% \and
% \IEEEauthorblockN{3\textsuperscript{rd} Given Name Surname}
% \IEEEauthorblockA{\textit{dept. name of organization (of Aff.)} \\
% \textit{name of organization (of Aff.)}\\
% City, Country \\
% email address or ORCID}
% \and
% \IEEEauthorblockN{4\textsuperscript{th} Given Name Surname}
% \IEEEauthorblockA{\textit{dept. name of organization (of Aff.)} \\
% \textit{name of organization (of Aff.)}\\
% City, Country \\
% email address or ORCID}
% \and
% \IEEEauthorblockN{5\textsuperscript{th} Given Name Surname}
% \IEEEauthorblockA{\textit{dept. name of organization (of Aff.)} \\
% \textit{name of organization (of Aff.)}\\
% City, Country \\
% email address or ORCID}
% \and
% \IEEEauthorblockN{6\textsuperscript{th} Given Name Surname}
% \IEEEauthorblockA{\textit{dept. name of organization (of Aff.)} \\
% \textit{name of organization (of Aff.)}\\
% City, Country \\
% email address or ORCID}
% }

\maketitle

\begin{abstract}
This document is a model and instructions for \LaTeX.
This and the IEEEtran.cls file define the components of your paper [title, text, heads, etc.]. *CRITICAL: Do Not Use Symbols, Special Characters, Footnotes, or Math in Paper Title or Abstract.
\lipsum[2]
\end{abstract}

\begin{IEEEkeywords}
component, formatting, style, styling, insert
\end{IEEEkeywords}

\section{Introduction}
This document is a model and instructions for \LaTeX.
Please observe the conference page limits. 
\lipsum[1]

\section{Diversity and Inclusivity of AIR4children with Non-tradional Education}

\subsection{Non-tradional education and schools}
Non-tradional education such as Montesory, etc.

\subsection{Montesory education}
Elkin et al. in 2014 explored the how robots can be used in the Montessori curriculum \cite{elkin2014}.
Authors conclude that the confidence and experience in robotics is crucial to deliver and communicate the right experince to encourgage students \cite{elkin2014}.
Similry Elkin et al. posed the question on the revision of new curriculums of technology that do not deviate from the purpose of the Montesory classrom \cite{elkin2014}.
Drigas and Gkeka in 2016 reviewed the application of information and communication technologes in the Montesory Method \cite{DrigasGkeka2016}.
Drigas and Gkeka mentioned the Manipulatives, as objects to develop motor skills or understan mathematical abstractions, are based on cultural areas, language, mathemaics and sensorial but little to none on technolgoical areas.
Drigas and Gkeka reviewed Montesory materials of the 21st century where interactive systems with sounds and lights, touch application to enhace visual literacy or the development of compuitational thinking and contrictuons of the physical world \cite{DrigasGkeka2016}
These indicate that the incorporation of such manipulatives with the use of robotics might led to reach scneraios to explore motor skill develipment, visualisation and computational thikning. 

Recenlty, Scippo and Ardolino reported a longitudinal study of the use of computationla thinking in five years participants of primary shcool in a Montessori school \cite{ScippoArdolino2021}
Scippo and Ardolino pointed out the importance of aligment of the Montessori material with the computationla thinking acitivites. 

\subsection{Other alternative education styles}
Number equations consecutively. To make your 
equations more compact, you may use the solidus (~/~), the exp function, or 
appropriate exponents. Italicize Roman symbols for quantities and variables, 
but not Greek symbols. Use a long dash rather than a hyphen for a minus 
sign. Punctuate equations with commas or periods when they are part of a 
sentence, as in:
\begin{equation}
a+b=\gamma\label{eq}
\end{equation}

Be sure that the 
symbols in your equation have been defined before or immediately following 
the equation. Use ``\eqref{eq}'', not ``Eq.~\eqref{eq}'' or ``equation \eqref{eq}'', except at 
the beginning of a sentence: ``Equation \eqref{eq} is . . .''


\section{Diversity and Inclusion with Open-Source AIR}

% \subsection{Some Common Mistakes}\label{SCM}
Some Common Mistakes
\begin{itemize}
\item The word ``data'' is plural, not singular.
\item In your paper title, if the words ``that uses'' can accurately replace the word ``using'', capitalize the ``u''; if not, keep using lower-cased.
\item Be aware of the different meanings of the homophones ``affect'' and ``effect'', ``complement'' and ``compliment'', ``discreet'' and ``discrete'', ``principal'' and ``principle''.
\item Do not confuse ``imply'' and ``infer''.
\item The prefix ``non'' is not a word; it should be joined to the word it modifies, usually without a hyphen.
\item There is no period after the ``et'' in the Latin abbreviation ``et al.''.
\item The abbreviation ``i.e.'' means ``that is'', and the abbreviation ``e.g.'' means ``for example''.
\end{itemize}

% \subsection{Figures and Tables}
Figures and Tables
\paragraph{Positioning Figures and Tables} Place figures and tables at the top and 
bottom of columns. Avoid placing them in the middle of columns. Large 
figures and tables may span across both columns. Figure captions should be 
below the figures; table heads should appear above the tables. Insert 
figures and tables after they are cited in the text. Use the abbreviation 
``Fig.~\ref{fig}'', even at the beginning of a sentence.

\begin{table}[htbp]
\caption{Table Type Styles}
\begin{center}
\begin{tabular}{|c|c|c|c|}
\hline
\textbf{Table}&\multicolumn{3}{|c|}{\textbf{Table Column Head}} \\
\cline{2-4} 
\textbf{Head} & \textbf{\textit{Table column subhead}}& \textbf{\textit{Subhead}}& \textbf{\textit{Subhead}} \\
\hline
copy& More table copy$^{\mathrm{a}}$& &  \\
\hline
\multicolumn{4}{l}{$^{\mathrm{a}}$Sample of a Table footnote.}
\end{tabular}
\label{tab1}
\end{center}
\end{table}

\begin{figure}[htbp]
\centerline{\includegraphics{fig1.png}}
\caption{Example of a figure caption.}
\label{fig}
\end{figure}

Figure Labels: Use 8 point Times New Roman for Figure labels. Use words 
rather than symbols or abbreviations when writing Figure axis labels to 
avoid confusing the reader. As an example, write the quantity 
``Magnetization'', or ``Magnetization, M'', not just ``M''. If including 
units in the label, present them within parentheses. Do not label axes only 
with units. In the example, write ``Magnetization (A/m)'' or ``Magnetization 
\{A[m(1)]\}'', not just ``A/m''. Do not label axes with a ratio of 
quantities and units. For example, write ``Temperature (K)'', not 
``Temperature/K''.

\section{Designing Inclusive Workshops}
\lipsum[1]

\section{Piloting Inclusive Workshops}
\lipsum[1]

\section{Conclusions and future work}
\lipsum[1]


\section*{Acknowledgment}

The preferred spelling of the word ``acknowledgment'' in America is without 
an ``e'' after the ``g''. Avoid the stilted expression ``one of us (R. B. 
G.) thanks $\ldots$''. Instead, try ``R. B. G. thanks$\ldots$''. Put sponsor 
acknowledgments in the unnumbered footnote on the first page.

\section*{References}

Please number citations consecutively within brackets \cite{pearce2013open}. The 
sentence punctuation follows the bracket \cite{tarik2017}. Refer simply to the reference 
number, as in \cite{UNICEF2020}---do not use ``Ref. \cite{hu2018}'' or ``reference \cite{hu2018}'' except at 
the beginning of a sentence: ``Reference \cite{hu2018} was the first $\ldots$''

Number footnotes separately in superscripts. Place the actual footnote at 
the bottom of the column in which it was cited. Do not put footnotes in the 
abstract or reference list. Use letters for table footnotes.

Unless there are six authors or more give all authors' names; do not use 
``et al.''. Papers that have not been published, even if they have been 
submitted for publication, should be cited as ``unpublished'' \cite{opensource2021}. Papers 
that have been accepted for publication should be cited as ``in press'' \cite{opensource2021}. 
Capitalize only the first word in a paper title, except for proper nouns and 
element symbols.

For papers published in translation journals, please give the English 
citation first, followed by the original foreign-language citation \cite{Serholt:2017}.

\bibliographystyle{IEEEtran}
\bibliography{../references/references}

\end{document}
